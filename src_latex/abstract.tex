% As a general rule, do not put math, special symbols or citations
% in the abstract
\begin{abstract}

JavaScript is a powerful scripting programming language that has gained a lot of attention this past decade. Initially used exclusively for client-side web development, it has evolved to become one of the most popular programming languages, with developers now using it for both client-side and server-side application development. Similar to applications written in other programming languages, JavaScript applications contain \emph{code smells}, which are \emph{poor} design choices that can negatively impact the quality of an application.
In this paper, we investigate code smells in JavaScript server-side applications with the aim to understand how they impact the fault-proneness {\color{blue}and the vulnerability} of applications, {\color{blue}and how they survive all along the projects}.
We detect 12 types of code smells in {\color{blue}1807} releases of {\color{blue}fifteen} popular JavaScript applications (\ie{} express, grunt, bower, less.js, request, {\color{blue}jquery, vue, ramda, leaflet, hexo, chart, webpack, webtorrent, moment, and riot}) and perform survival analysis, comparing the time until a fault occurrence, in files containing code smells and files without code smells. {\color{blue}We then do the same survival analysis, but with a line grain approach (wich means considering the lines where the code smells and the potential bugs appear), and with a line grain approach including dependencies (which means considering the lines where functions, objects, variables are called). We also perform file grain, line grain, and line grain including dependencies survival analysis, comparing the time until a vulnerability appears. Finally, we perform a survival analysis on code smells to know how long they survive.} Results show that (1) on average, files without code smells have hazard rates {\color{blue}76\%} lower than files with code smells {\color{blue}in our file grain analysis, 20\% lower in our line grain analysis, and 38\% lower in our line grain analysis considering dependencies.} (2) Among the studied smells, ``Variable Re-assign", ``Assignment In Conditional statements", and ``Complex Code" smells have the highest fault hazard rates. {\color{blue}(3) Files without code smells are not necessarely less vulnerable than files with code smells, but this conclusion needs to be mitigated because of the weaknesses of our vulnerability database. (4) Among the studied smells, ``Variable Re-assign" and ``This Assign" have the highest vulnerability hazard rates. (5) Code smells, and particularly ``Variable Re-assign", tend to be created at the file creation, are not enough removed from the system, and have a high chance of surviving a very long time after their introduction; ``Variable Re-assign" is also the most proliferated code smells.} Additionally, we conduct a survey with 1,484 JavaScript developers, to understand the perception of developers towards our studied code smells. We found that developers consider ``Nested Callbacks", ``Variable Re-assign" and ``Long Parameter List" code smells to be serious design problems that hinder the maintainability and reliability of applications. This assessment is in line with the findings of our quantitative analysis.
Overall, code smells affect negatively the quality of JavaScript applications and developers should consider tracking and removing them early on before the release of applications to the public.

\end{abstract} 