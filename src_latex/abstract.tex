% As a general rule, do not put math, special symbols or citations
% in the abstract
\begin{abstract}

JavaScript is a powerful scripting programming language that has gained a lot of attention this past decade. Initially used exclusively for client-side web development, it has evolved to become one of the most popular programming languages, with developers now using it for both client-side and server-side application development. Similar to applications written in other programming languages, JavaScript applications contain \emph{code smells}, which are \emph{poor} design choices that can negatively impact the quality of an application.
In this paper, {\color{blue}we extend the work of Amir Saboury et al.~\cite{saboury2017empirical} by investigating} code smells in {\color{blue}more} JavaScript server-side applications with the aim to understand how they impact the fault-proneness of applications, {\color{blue}and how they survive all along the projects}.
We detect 12 types of code smells in {\color{blue}1807} releases of {\color{blue}fifteen} popular JavaScript applications (\ie{} express, grunt, bower, less.js, request, {\color{blue}jquery, vue, ramda, leaflet, hexo, chart, webpack, webtorrent, moment, and riot}) and perform survival analysis, comparing the time until a fault occurrence, in files containing code smells and files without code smells. {\color{blue}In a different way than our predecessors, we do the survival analysis with a line grain approach (wich means considering the lines where the code smells and the potential bugs appear), and with a line grain approach including dependencies (which means considering the lines where functions, objects, variables are called). Finally, we perform a survival analysis on code smells to know how long they survive.} Results show that (1) on average, files without code smells have hazard rates {\color{blue}20\%} lower than files with code smells {\color{blue}in our line grain analysis, and 38\% lower in our line grain analysis considering dependencies.} (2) Among the studied smells, ``Variable Re-assign", ``Assignment In Conditional statements", and ``Complex Code" smells have the highest fault hazard rates. {\color{blue}(3) Code smells, and particularly ``Variable Re-assign", tend to be created at the file creation, are not enough removed from the system, and have a high chance of surviving a very long time after their introduction; ``Variable Re-assign" is also the most proliferated code smells.} 
Overall, code smells affect negatively the quality of JavaScript applications and developers should consider tracking and removing them early on before the release of applications to the public.

\end{abstract} 